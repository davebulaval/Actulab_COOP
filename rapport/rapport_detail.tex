\documentclass[11pt,french]{report}\usepackage[]{graphicx}\usepackage[]{color}
%% maxwidth is the original width if it is less than linewidth
%% otherwise use linewidth (to make sure the graphics do not exceed the margin)
\makeatletter
\def\maxwidth{ %
  \ifdim\Gin@nat@width>\linewidth
    \linewidth
  \else
    \Gin@nat@width
  \fi
}
\makeatother

\definecolor{fgcolor}{rgb}{0.345, 0.345, 0.345}
\newcommand{\hlnum}[1]{\textcolor[rgb]{0.686,0.059,0.569}{#1}}%
\newcommand{\hlstr}[1]{\textcolor[rgb]{0.192,0.494,0.8}{#1}}%
\newcommand{\hlcom}[1]{\textcolor[rgb]{0.678,0.584,0.686}{\textit{#1}}}%
\newcommand{\hlopt}[1]{\textcolor[rgb]{0,0,0}{#1}}%
\newcommand{\hlstd}[1]{\textcolor[rgb]{0.345,0.345,0.345}{#1}}%
\newcommand{\hlkwa}[1]{\textcolor[rgb]{0.161,0.373,0.58}{\textbf{#1}}}%
\newcommand{\hlkwb}[1]{\textcolor[rgb]{0.69,0.353,0.396}{#1}}%
\newcommand{\hlkwc}[1]{\textcolor[rgb]{0.333,0.667,0.333}{#1}}%
\newcommand{\hlkwd}[1]{\textcolor[rgb]{0.737,0.353,0.396}{\textbf{#1}}}%
\let\hlipl\hlkwb

\usepackage{framed}
\makeatletter
\newenvironment{kframe}{%
 \def\at@end@of@kframe{}%
 \ifinner\ifhmode%
  \def\at@end@of@kframe{\end{minipage}}%
  \begin{minipage}{\columnwidth}%
 \fi\fi%
 \def\FrameCommand##1{\hskip\@totalleftmargin \hskip-\fboxsep
 \colorbox{shadecolor}{##1}\hskip-\fboxsep
     % There is no \\@totalrightmargin, so:
     \hskip-\linewidth \hskip-\@totalleftmargin \hskip\columnwidth}%
 \MakeFramed {\advance\hsize-\width
   \@totalleftmargin\z@ \linewidth\hsize
   \@setminipage}}%
 {\par\unskip\endMakeFramed%
 \at@end@of@kframe}
\makeatother

\definecolor{shadecolor}{rgb}{.97, .97, .97}
\definecolor{messagecolor}{rgb}{0, 0, 0}
\definecolor{warningcolor}{rgb}{1, 0, 1}
\definecolor{errorcolor}{rgb}{1, 0, 0}
\newenvironment{knitrout}{}{} % an empty environment to be redefined in TeX

\usepackage{alltt}
  \usepackage{babel} %%french
  \usepackage{amsmath,amsfonts,amssymb} %%maths
  \usepackage[utf8]{inputenc}   % LaTeX
  \usepackage[T1]{fontenc}      % LaTeX
  \usepackage[dvipsnames,table,xcdraw]{xcolor}
  
  %% image
  \usepackage{graphicx}
  	\graphicspath{{./fig/}} %fig path
  %%profondeur de la numérotation du TOC
  \setcounter{secnumdepth}{3} 
  %%lien hypertex
  \usepackage[colorlinks, allcolors=Blue]{hyperref} 
  %% package pour modifier les chapitres #2
  \usepackage{titlesec}
  %% for number separation
  \usepackage[autolanguage]{numprint}

%% for licence
  \usepackage{tabularx}
  \usepackage{multirow}
  %% no indent
  \setlength{\parindent}{0pt}
  \frenchbsetup{StandardItemLabels=true} % pour obtenir des puces par défaut dans les listes à puces A.B.
  %% for moreInfo 
  \usepackage{fontawesome} %%also cool logo
  \usepackage[framemethod=TikZ]{mdframed}
  
  % Creation of environment to add additional informations
  % Provided by Samuel Cabral Cruz
\mdfsetup{
	linewidth=2pt,
	nobreak=true,
	backgroundcolor=Blue!10,
	roundcorner=10pt}	
\newenvironment{moreInfo}[1]
	{\begin{mdframed}
	\textcolor{darkgray}{\huge \raisebox{-3.5pt}{\faInfo} 
	\hspace{0.5cm} \large\bfseries #1}\\[5pt]
	\normalsize
	\makebox[0.1\textwidth][l]{}	
	\begin{minipage}{10cm}}
	{	\end{minipage}
	\end{mdframed}}
	
%% meta donnée document
\title{Résolution de problématique de \\ Co-Operators \\ \bigskip Où sont les clients que nous ciblons ?}
\author{\textbf{Présenter par \\ David Beauchemin}}
\date{\today}
\def\versionnumber{1.0}
\IfFileExists{upquote.sty}{\usepackage{upquote}}{}
\begin{document}



\makeatletter
  \begin{titlepage}
  \centering
      {\LARGE \textbf{\textsc{Actulab}}}\\
    \vspace{2cm}
    \vspace{2cm}
      {\LARGE \textbf{\@title}} \\
    \vspace{2cm}
    \vfill
       {\Large \@author} \\
    \vspace{8cm}
        {\large\textbf{\versionnumber}}\\
    \vfill
  \end{titlepage}
\makeatother

%%%%license %%%%%
\include{section/license}
\pagebreak

%%%%Remerciements %%%%%
%% remerciements

\subsubsection*{Remerciements}

\includegraphics[height=2cm, width = 3cm]{fig/signature.png}




\tableofcontents

\newpage

%%%%%% SOMMAIRE %%%%%%%%
\chapter*{Rapport sommaire}

\section*{Détails techniques sommaires}
\begin{itemize}
\item Données libre du recensement canadien de \href{http://www12.statcan.gc.ca/census-recensement/2016/dp-pd/prof/details/download-telecharger/comp/page_dl-tc.cfm?Lang=F}{2016}  
\end{itemize}
\section*{Hypothèse utilisée}

\section*{Application \emph{Shiny}}

\section*{Résultat sommaire}
\href{https://davebulaval.shinyapps.io/personnasIdentificateur/}{Shiny}
\section*{Code source}
Il est possible de consulter l'ensemble du code source et les données utilisées pour le projet à partir de la \href{https://davebulaval.github.io/Actulab_COOP/}{page web} du dépôt \href{https://github.com/davebulaval/Actulab_COOP}{\emph{GitHub}}.



%%%%%% Content %%%%%%%%
\chapter*{Rapport détaillé}

\section*{Détails de la problématique}
Co-operators désire déterminer la distribution géographique de différents personas à l'aide de données ouvertes. 
...

\section*{Détails techniques}
Toutes les données utiliser dans le cadre de se projet provienne de source libre. 

\section*{Objectifs}
...

\section*{Outils utilisés}

\section*{Méthodologie}
Indépendance
test chi square

\section*{Avantage-Inconvénient}


\section*{Résultats}

\section*{Piste d'amélioration}

\section*{Conclusion}



\end{document}
